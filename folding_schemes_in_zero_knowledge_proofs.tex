\documentclass[12pt]{article}
\usepackage[utf8]{inputenc}
\usepackage{hyperref}
\usepackage{amsmath}

\title{Understanding Zero-Knowledge Proofs: A Simple Introduction to Folding Schemes}
\author{Cyprian Omukhwaya Sakwa}
\date{\today}

\begin{document}
	
	\maketitle
	
	\section*{Introduction}
	
	Zero-knowledge proofs (ZKPs) are cryptographic protocols that enable one party (the prover) to convince another (the verifier) of a statement's validity without revealing any additional information. These protocols are fundamental in applications like privacy-preserving blockchains and secure multiparty computations. This essay will focus on a subdomain of ZKPs: \textbf{folding schemes}, which enable efficient and scalable proof systems.
	
	Folding schemes compress multiple computational statements into fewer, simpler ones. This mechanism reduces the complexity for verifiers while maintaining the integrity of the proofs. 
	
	\section*{The Need for Folding Schemes}
	
	In cryptography, proving complex computational statements is resource-intensive. Consider verifying the validity of a large dataset processed by a blockchain smart contract. Traditional approaches involve executing every operation, which can be slow and expensive. Folding schemes solve this problem by collapsing proofs incrementally.
	
	For instance, the Nova protocol employs folding to recursively verify multiple computations over time. At every step, computations are folded into a compact representation, ensuring the verifier’s effort remains constant regardless of the number of statements.
	
	\section*{How Folding Schemes Work}
	
	At its core, folding relies on two principles:
	
	\paragraph*{1. Instance Folding}
	Combine two computational statements \( S_1 \) and \( S_2 \) into a single statement \( S_3 \). This merging ensures that verifying \( S_3 \) implies the correctness of \( S_1 \) and \( S_2 \).
	
	\paragraph*{2. Constraint Folding}
	Aggregate the logical constraints governing \( S_1 \) and \( S_2 \). This step ensures that proving \( S_3 \) satisfies all combined constraints.
	
	\subsubsection*{A Simple Analogy}
	Imagine two jigsaw puzzles, \( P_1 \) and \( P_2 \), each representing a complex computation. A folding scheme combines these into a smaller puzzle, \( P_3 \), such that solving \( P_3 \) ensures \( P_1 \) and \( P_2 \) are correctly assembled. The original puzzles no longer need verification individually.
	
	\section*{Applications in Ethereum}
	
	Ethereum’s scalability and security challenges make ZKP-based folding schemes crucial. For example:
	
	\begin{enumerate}
		\item \textbf{Scalable Rollups}: Folding schemes can compress transaction proofs in rollups like zkSync or StarkNet, reducing on-chain verification costs.
		\item \textbf{Recursive Proof Composition}: Protocols like Halo employ folding to recursively verify smart contract executions, enabling scalability without compromising security.
	\end{enumerate}
	
	\section*{Conclusion}
	
	Folding schemes are a cornerstone of modern ZKP systems, enabling scalability, efficiency, and quantum resilience. Whether enhancing Ethereum’s rollup solutions or advancing privacy-preserving applications, folding schemes ensure robust cryptographic proofs in resource-constrained environments. By focusing on practical applications, we can unlock blockchain's full potential.
	
\end{document}
